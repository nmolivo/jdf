\documentclass[
	%a4paper, % Use A4 paper size
	letterpaper, % Use US letter paper size
]{jdf}

\addbibresource{references.bib}

\author{Natalie Olivo}
\email{nolivo3@gatech.edu}
\title{Representing knowledge as states and transfers}

\begin{document}
%\lsstyle

\maketitle

\begin{Abstract}
	This paper covers one attempt to abstract the Lord of The Rings characters crossing a river with The Ring problem as a Semantic Network. Further discussion around its completeness, and efficiency will follow.
\end{abstract}

\section{The Lord of The Rings Problem}
Four entities must cross a river, under a number of constraints. 

% For example in Figure \ref{fig:Palatino} on page \pageref{fig:Palatino}...

\subsection{Semantic Network}
The semantic network in question must Lexically capture objects, relationships, and potentially link labels. Structurally, it must capture directional links, and Procedurally, must  show legal procedures. 

\begin{jdffigure}
	\includegraphics[height=6cm]{Figures/semantic_network_1.png}%
	\captionof{figure}{The blue part represents the river, the green parts represent either side of the river. The brown rhombus represents a two-person boat. And finally, the black arrow represents a transition from one state to another.}\label{fig:flowchart}%
	\end{jdffigure}

###
\begin{itemize}
	\item \includegraphics[scale=0.7]{Figures/gollum.png} Gollum
	\item \includegraphics[scale=0.7]{Figures/frodo.png} Frodo
	\item \includegraphics[scale=0.7]{Figures/sam.png} Sam
	\item \includegraphics[scale=0.7]{Figures/ring.png} Ring
\end{itemize}

\subsection{Additional constraints}
Transition and rules governing the relationships are not immediately obvious in a semantic network representing two states with a transition between them.

Transitions can be described as reliant on a two-person boat which alternates between left and right banks. Sam \includegraphics[scale=0.7]{Figures/sam.png} must take up one place in the boat for each transition because he's the only person capable of maneuvering the boat. Procedurally, all transactions that undo the previous transaction, are invalid. And finally, all transactions which result back to the initial state are invalid as they indicate an inefficiency.

Frodo \includegraphics[scale=0.7]{Figures/frodo.png} and Gollum  \includegraphics[scale=0.7]{Figures/gollum.png} cannot be in a boat with The Ring, or on the same side of the river as The Ring  \includegraphics[scale=0.7]{Figures/ring.png} without Sam  \includegraphics[scale=0.7]{Figures/sam.png} there. Neither Gollum or Frodo can be alone with The Ring at any time. 

\section{Generate and Test}
One approach to solve this puzzle is to use the Generate and Test method. This consists of proposing all possible moves, and then testing that each of them are valid. 

\subsection{Smart Generator}
For our LOTR example we're going to use a smart generator that presents all valid transitions. It will understand valid transitions as:
\begin{itemize}
\item Only preserving moves of items that exist on the side of the bank the boat is currently on. 
\item Only generating moves where Sam \includegraphics[scale=0.7]{Figures/sam.png} is always on the boat, since he's the only person who can drive it. 
\item Finally, our generator will not generate a move that undoes the prior move.
\end{itemize}

\subsection{Tester}
Our smart tester will then validate that The Ring \includegraphics[scale=0.7]{Figures/ring.png} is not alone with either or both Gollum \includegraphics[scale=0.7]{Figures/gollum.png} and Frodo \includegraphics[scale=0.7]{Figures/frodo.png} at any time.

\section{Full Semantic Network}
Given the Generate and Test methods described above, the full semantic network will look like this: 

Tester behavior key:
\begin{itemize}
	\item \includegraphics[scale=0.7]{Figures/valid.png} Valid
	\item \includegraphics[scale=0.7]{Figures/invalid.png} Invalid
\end{itemize}

\begin{jdffigure}
\includegraphics[height=6cm]{Figures/flowchart.png}%
\captionof{figure}{Make sure your flowcharts are more useful than this one. Source: \href{https://xkcd.com/1195/}{XKCD}.}\label{fig:flowchart}%
\end{jdffigure}

Figure captions should be placed beneath the corresponding figure, indented 1" on the left and right sides. The label for the figure, e.g. "Figure 1," should be set in bold italics followed by an em dash, and the entire caption should be 8.5 points with 14 points of line spacing. Again, Word and LaTeX will number these automatically using the \emph{Figure Caption} paragraph style, but Docs users will need to number these manually. If need be, you may have one figure caption corresponding to multiple consecutive figures and use either locational descriptors (e.g. "top left," "middle") or labels (e.g. "A", "B") to map parts of the caption to parts of the figure. Make sure that caption falls on the same page as the corresponding figure or table; you may rearrange text to make this work.

In Microsoft Word, you may need to either change the image’s text wrap settings to "Top and Bottom" or change the line spacing of the image to 1.0.

\subsection{Tables}
You have freedom to format tables in the way that works best for your data. Generally, text should be left-aligned and numbers should be right-aligned or aligned at the decimal – you can do this using a \href{https://practicaltypography.com/tabs-and-tab-stops.html}{custom tab stop}. The default table style (shown below) reduces the text size to be equal to the caption text.

Table captions should be formatted the same way as figure captions, but they should be placed above the table. The popular mnemonic for this is: figures at the foot, tables at the top. Like figures, tables should not exceed the margins and should be centered on the page.

\begin{jdftable}
\captionof{table}{Mathematical constants. Notice how the approximations align at the decimal.}\label{table:Example}
\small % Reduce font size
\begin{tabular}{@{} L{0.15\linewidth} c S L{0.52\linewidth}}
	\textbf{Name} & \textbf{Symbol} & \textbf{Approximation} & \textbf{Description} \\
	\toprule[0.5pt]
	Golden ratio & $\phi$ & 1.618 & Number such that the ratio of " to the number is equal to the ratio of its reciprocal to 1\\
	\midrule
	Euler's number & $e$ & 2.71828 & Exponential growth constant\\
	\midrule
	Archimedes' constant & $\pi$ & 3.14 & The ratio between circumference and diameter of a circle\\
	\midrule
	One hundred & A+ & 100.00 & The grade we hope you’ll all earn in this class\\
\end{tabular}
\end{jdftable}

\subsection{Additional elements}
There are additional elements you may want to include in your paper, such as in-line or block quotes, lists, and more. For other content types not covered here, you have flexibility in determining how it should be used in this format.

\subsubsection{Quotes}
If you would like to quote an outside source, you may do so with quotation marks followed by a citation. If a quote is fewer than three lines, you may write it in-line. It is acceptable to replace pronouns with their target in brackets for clarity. For example, "Heavy use of peer grading would compromise [the school’s] reputation" \citep{joyner2016}. If a quote exceeds three lines, you should set it as its own paragraph with 0.5" side margins, using the \emph{Blockquote} style.

\begin{quotation}
"Whether or not the grades generated by peers are reliably similar to grades generated by experts is only one factor worth considering, however. Student perception is also an important factor. A recent study indicated that reliance on peer grading is one of the top drivers of high MOOC dropout rates. This problem may be addressed by reintroducing some expert grading where possible." \citep{joyner2016}
\end{quotation}

\subsubsection{Lists}
Bulleted and numbered lists are indented 0.5" from the left margin, with the bullet or number hanging in the margin by 0.25" (the default format).

Bullet points:

\begin{itemize}
	\item First bullet point item
	\item Second bullet point item
\end{itemize}

Numbered list:

\begin{enumerate}
	\item First numbered item
	\item Second numbered item
\end{enumerate}

\section{Procedural elements}
\subsection{In-line citations}
Articles or sources to which you refer should be cited in-line with the authors’ names and the year of publication.\footnote{In-line citations are preferred over footnotes, and we favor APA citation format for both in-line citations and reference lists. Refer to the \href{https://owl.purdue.edu/owl/research_and_citation/apa_style/apa_formatting_and_style_guide/in_text_citations_the_basics.html}{Purdue Online Writing Lab}, or follow the above examples. Footnotes should use 8.5 point text with 1.26 line spacing.} The citation should be placed close in the text to the actual claim, not merely at the end of the paragraph. For example: students in the OMSCS program are older and more likely to be employed than students in the on-campus program \citep{joyner2017}. In the event of multiple authors, list them. For example: research finds sentiment analysis of the text of OMSCS reviews corresponds to student-assigned ratings of the course \citep{newman2018}. You may also cite multiple studies together. For example: several studies have found students in the online version of an undergraduate CS1 class performed equally with students in a traditional version (\cite{joyner2018a}; \cite{joyner2018b}). If you would like to refer to an author in text, you may also do so by including the year (in parentheses) after the author’s name in the text. If a publication has more than 4 authors, you may list the first author followed by ‘et al.’ For example: \citeinl{joyner2016} claim that a round of peer review prior to grading may improve graders’ efficiency and the quality of feedback given. This applies to parenthetical citations as well, e.g. \citep{joyner2016}.

\subsection{Reference lists}
References should be placed at the end of the paper in a dedicated section. Reference lists should be numbered and organized alphabetically by first author’s last name. If multiple papers have the same author(s) and year, you may append a letter to the end of the year to allow differentiated in-line text (e.g. Joyner, 2018a and Joyner, 2018b in the section above). If multiple papers have the same author(s), list them in chronological order starting with the older paper. Only works that are cited in-line should be included in the reference list. The reference list does not count against the length requirements.

\section{References}
\printbibliography[heading=none]

\section{Appendices}
You may optionally move certain information to appendices at the end of your paper, after the reference list. If you have multiple appendices, you should create a section with a \emph{Heading 1} of "Appendices." Each appendix should begin with a descriptive \emph{Heading 2}; appendices can thus be referenced in the body text using their heading number and description, e.g. "Appendix 5.1: Survey responses." If you have only one appendix, you can label it with the word "Appendix" followed by a descriptive title, e.g., "Appendix: Survey responses."

These appendices do not count against the page limit, but they should not contain any information required to answer the question in full. The body text should be sufficient to answer the question, and the appendices should be included only for you to reference or to give additional context. If you decide to move content to an appendix, be sure to summarize the content and note it in relevant place in the body text, e.g., "The raw data can be viewed in \emph{Appendix 5.1: Survey responses}."

\end{document}
